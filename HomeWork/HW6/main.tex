\documentclass{article}
\usepackage[utf8]{inputenc}
\usepackage{pgfplots,multicol}
\usepackage{tikz-qtree}
\usepackage{url}
\usepackage{hyperref}
\usepackage{xcolor}
\usepackage{avm}
\usepackage{rtrees}
\usepackage{forest}
\useforestlibrary{linguistics}
\usepackage{rotating, graphicx}
\newcommand{\comment}[1]{}
\forestapplylibrarydefaults{linguistics}
\hypersetup{
  colorlinks   = true, %Colours links instead of ugly boxes
  urlcolor     = red, %Colour for external hyperlinks
  linkcolor    = blue, %Colour of internal links
  citecolor   = blue %Colour of citations
}
\pgfplotsset{compat=newest} 
\mathchardef\period=\mathcode`.
\newcommand{\textarray}[1]{\ensuremath{\left[ \mbox{\ttfamily\begin{tabular}{l} #1 \end{tabular}}\right]}}
\begin{document}
\title{566 HW6}
\author{Daniel Campos  \tt {dacampos@uw.edu}}
\date{11/15/2019}
\maketitle 
\section{Chapter 11, Problem 3}
\subsection{RESTR of: That Dana is smoking annoys Leslie}
\begin{avm}RESTR & \< \avml \[RELN & {\it annoy} \\ SIT & {\it s1} \\ANNOYANCE & {\it s2}\\ANNOYED & {\it j}\], \[RELN & {\it name} \\ NAME & {\it Dana} \\ NAMED & {\it i}  \], \[RELN & {\it name} \\ NAME & {\it Leslie} \\ NAMED & {\it j} \],  \[RELN & {\it smoke} \\ SIT & {\it s2} \\ SMOKER & {\it i} \] \avmr \>\end{avm}
\subsection{Tree}
\scalebox{1}{
\begin{turn}{0}
\begin{forest}
[S_{s1} [CP_{s2} [C_{s2} [That] ] [S_{s2} [NP_i [Dana] ] [VP_{s2} [V [is] ] [V_{s2} [smoking] ] ] ] ] [VP_{s1} [V_{s1} [annoys]] [NP_j [Leslie] ] ] ] ]
\end{forest} \end{turn} }
\subsection{Explain SIT value of smoke predication gets identified with annoyance value of the annoy}
1. The lexical entry for annoy identifies the annoyance role of the annoy predication on its RESTR list with the INDEX value of first element on its ARG-ST list. \\
2. The lexical entry for smoke identifies the SIT value of the smoke predication in its RESTR list with its own index, {\it s2}.\\
3. The Argument Realization Principle(ARP) (together with the SHAC) identifies the first element of the ARG-ST list of annoy with its SPR value.\\
4. Via the Head Specifier Rule(HSR) and the Head Feature Principle(HFP) the SPR of is $VP_{s1}$ identified with the SPR of $V_{s1}$.\\
5. The HSR identifies $CP_{s2}$ as the SPR of $VP_{s1}$.\\
6. The Semantic Inheritance Principle(SIP) identifies the index of the mother $VP_{s2}$ element that licenses {\it is smoking} with the index of its head daughter $V_{s2}$. \\
7. $S_{s2}$, via the SIP and the HSR, identifies its index with that of the head daughter of $VP_{s2}$. \\
8. The HCR identifies $S_{s2}$ as the only element in COMPS list for the complimentizer that licenses {\it that}. \\
9. The constraint on the type {\it comp-lxm}, which {\it that} is a sub type of, identifies its index with the index of the sole element in its AGR-ST, which is a $S_{s2}$ node. \\
10. THE ARP identifies the S node in the AGR-ST list with the sole element in its COMPS list, $S_{s2}$. \\
11. Via the HCR and the SIP the CP that licences {\it That Dana is smoking} identifies its index with that of its head daughter which also makes its index {\it s2}. Thus the sit value of smoke, {\it s2} get identified with the annoyance value of annoy.
\section{Chapter 11, Problem 4}
\subsection{constraints imposed on lexical sequences from 3rd-sing verb lexical verb}
\begin{avm} \< \avml annoys , \[{\it 3sing-stv-lxm} \\SYN & \[HEAD & \[FORM & {\it fin}\\AGR & {\@2}{\it 3sing}  \\PRED & -\] \\ VAL & \[ SPR & \< \avml {\@3} \avmr \> \\ COMPS \< \avml {\@4} \avmr \> \\ MOD & \< \avml \avmr \>\] \] \\ARG-ST & \< \avml {\@3}\[SYN & \[HEAD & \[AGR & {\@2} \\ CASE & {\it nom} \] \] \\ SEM & \[INDEX & {\@1} \] \], {\@4}{NP}_i\avmr \> \\SEM & \[INDEX & {\it s} \\MODE & {\it prop} \\ RESTR & \< \avml \[RELN & {\it annoy} \\ SIT & s \\ANNOYANCE & {\@1} \\ ANNOYED & {\it i}  \]  \avmr \> \oplus \hfil ... \] \]\avmr \> \end{avm}
\subsection{Constraints imposed by applying Extraposition Lexical Rule to answer in A}
\begin{avm} \< \avml annoys , \[{\it 3sing-stv-lxm} \\SYN & \[HEAD & \[FORM & {\it fin}\\AGR & {\@2}{\it 3sing}  \\PRED & -\] \\ VAL & \[ SPR & \< \avml {\@5}NP \[FORM & {\it it} \] \avmr \> \\ COMPS \< \avml {\@4}, {\@3} \avmr \> \\ MOD & \< \avml \avmr \>\] \] \\ARG-ST & \< \avml {\@5,  {\@4}{NP}_i,{\@3}\[SYN & \[HEAD & \[AGR & {\@2} \\ CASE & {\it nom} \] \] \\ SEM & \[INDEX & {\@1} \] \],\avmr \> \\SEM & \[INDEX & {\it s} \\MODE & {\it prop} \\ RESTR & \< \avml \[RELN & {\it annoy} \\ SIT & s \\ANNOYANCE & {\@1} \\ ANNOYED & {\it i}  \]  \avmr \> \oplus \hfil ... \] \]\avmr \> \end{avm}
\subsection{Tree for:It annoys Lee that Fido barks } 
Rotated and scaled for ease of reading. \\ 
\\
\scalebox{.5}{
\begin{turn}{90}
\begin{forest}
[S  \\ \begin{avm} \[SEM & \[INDEX & s1 \\MODE & prop \\ REST & \< \avml \[RELN & {\it annoy} \\ SIT & {\it s1} \\ANNOYANCE & {\it s}\\ANNOYED & {\it j}\]  \[RELN & name \\ NAME & Lee \\ NAMED & j \]  \[RELN & bark \\ SIT & S \\ BARKER & i\]  \[RELN & name \\ NAME & Fido \\ NAMED & i \]   \avmr \> \] \]\end{avm}[NP  \\ \begin{avm}{\@1} \[SYN & \[VAL & \[SPR & \< \avml \avmr \> \\ COMPS & \< \avml  \avmr \> \] \] \\ SEM & \[MODE & non \\ REST & \< \avml  \avmr \> \] \]\end{avm} [It] ] [VP  \\ \begin{avm} \[SYN & \[VAL & \[SPR & \< \avml {\@1} \avmr \> \\ COMPS & \< \avml  \avmr \> \] \] \\ SEM & \[INDEX & s1 \\MODE & prop \\ REST & \< \avml \[RELN & {\it annoy} \\ SIT & {\it s1} \\ANNOYANCE & {\it s}\\ANNOYED & {\it j}\] \[RELN & name \\ NAME & Lee \\ NAMED & j \] \[RELN & bark \\ SIT & S \\ BARKER & i\]  \[RELN & name \\ NAME & Fido \\ NAMED & i \]  \avmr \> \] \]\end{avm} [V \\ \begin{avm} \[SYN & \[VAL & \[SPR & \< \avml {\@1} \avmr \> \\ COMPS & \< \avml {\@2} {\@3}  \avmr \> \] \] \\ SEM & \[INDEX & s1 \\MODE & prop \\ REST & \< \avml \[RELN & {\it annoy} \\ SIT & {\it s1} \\ANNOYANCE & {\it s}\\ANNOYED & {\it j}\]  \avmr \> \] \]\end{avm} [annoys] ] [NP \\ \begin{avm} {\@2} \[SYN & \[VAL & \[SPR & \< \avml\avmr \> \\ COMPS & \< \avml  \avmr \> \] \] \\ SEM & \[INDEX & j \\MODE & ref \\ REST & \< \avml \[RELN & name \\ NAME & Lee \\ NAMED & j \] \avmr \> \] \]\end{avm}  [Lee] ] [CP \\ \begin{avm} {\@3} \[SYN & \[VAL & \[SPR & \< \avml  \avmr \> \\ COMPS & \< \avml  \avmr \> \] \] \\ SEM & \[INDEX & s \\MODE & prop \\ REST & \< \avml  \avmr \> \] \]\end{avm} [C \\ \begin{avm} \[SYN & \[VAL & \[SPR & \< \avml \avmr \> \\ COMPS & \< \avml {\@4}  \avmr \> \] \] \\ SEM & \[INDEX & s \\MODE & prop \\ REST & \< \avml    \avmr \> \] \]\end{avm} [that] ] [S \\ \begin{avm} {\@4} \[SYN & \[VAL & \[SPR & \< \avml \avmr \> \\ COMPS & \< \avml  \avmr \> \] \] \\ SEM & \[INDEX & s \\MODE & prop \\ REST & \< \avml \[RELN & bark \\ SIT & S \\ BARKER & i\]  \[RELN & name \\ NAME & Fido \\ NAMED & i \] \avmr \> \] \]\end{avm} [NP \\ \begin{avm} {\@5}\[SYN & \[VAL & \[SPR & \< \avml \avmr \> \\ COMPS & \< \avml  \avmr \> \] \] \\ SEM & \[INDEX & \[RELN & bark \\ SIT & S \\ BARKER & i\] \\MODE & ref \\ REST & \< \avml \[RELN & name \\ NAME & Fido \\ NAMED & i \] \avmr \> \] \]\end{avm} [Fido] ][VP \\ \begin{avm} \[SYN & \[VAL & \[SPR & \< \avml {\@5} \avmr \> \\ COMPS & \< \avml  \avmr \> \] \] \\ SEM & \[INDEX & s \\MODE & prop \\ REST & \< \avml \[RELN & bark \\ SIT & S \\ BARKER & i\]  \avmr \> \] \]\end{avm} [barks] ] ] ] ] ]
\end{forest} \end{turn} }
\subsection{Why doesn't grammar license: It annoys me Sandy.}
The grammar licences iii and not iv because the Extraposition Lexical Rule requires that the input word feature have a SPR value that is a CP. In {\it it annoys me sandy} the sentence includes a dummy NP[FORM it] at the beginning but the structure following annoys is a NP instead of the expected CP. 
\section{Chapter 11, Problem 7}
i) It was assumed that the ad worked.
\subsection{Lexical Entry assume}
\begin{avm} \< \avml  assume , \[{\it stv-lxm}  \\SYN & \[HEAD & \[{\it verb} \\ AGR & {\@1} \]\] \\VAL & \[SPR & \< \avml \[AGR & {\@1}\]  \avmr \> \] \\ ARG-ST & \< \avml NP_j, X_i \avmr \> \\SEM & \[INDEX & {\it s} \\RESTR & \< \avml \[RELN & {\it assume} \\ SIT & s \\ASSUMER & {\it j} \\ ASSUMPTION & {\it i}  \]  \avmr \>\] \]\avmr \>\end{avm}
\subsection{What rules and what order must they be applied to make assume become passivized and extraposed.}
 Passive,Constant Lexeme,Extraposition Lexical Rule in that order. This is our order because Extraposition Lexical rule needs an input word, Passive and Constant needs a lexeme an and Constant makes a word. Any other order and it wouldnt work. 
\subsection{Output value of each of the lexical rules}
 Input is (1), output of Passive is (2), Output of Constant Lexeme is (3), Output of Extraposition Lexical Rule is (4)
\begin{equation}
\begin{avm} \< \avml  assume , \[{\it stv-lxm}  \\SYN & \[HEAD & \[{\it verb} \\ AGR & {\@1} \]\] \\VAL & \[SPR & \< \avml \[AGR & {\@1}\]  \avmr \> \] \\ ARG-ST & \< \avml NP_j, X_i \avmr \> \\SEM & \[INDEX & {\it s} \\RESTR & \< \avml \[RELN & {\it assume} \\ SIT & s \\ASSUMER & {\it j} \\ ASSUMPTION & {\it i}  \]  \avmr \>\] \]\avmr \>\end{avm}
\end{equation}
\begin{equation}
\begin{avm} \< \avml  assumed , \[{\it part-lxm}  \\SYN & \[HEAD & \[{\it verb} \\ AGR & {\@1} \\ FORM & {\it pass} \]\] \\VAL & \[SPR & \< \avml \[AGR & {\@1}\]  \avmr \> \] \\ ARG-ST & \< \avml X_i, (\[PP \\ FORM & by \\ INDEX & j \])\avmr \> \\SEM & \[INDEX & {\it s} \\RESTR & \< \avml \[RELN & {\it assume} \\ SIT & s \\ASSUMER & {\it j} \\ ASSUMPTION & {\it i}  \]  \avmr \>\] \]\avmr \>\end{avm}
\end{equation}
\begin{equation}
\begin{avm} \< \avml  assumed , \[{\it word}  \\SYN & \[HEAD & \[{\it verb} \\ AGR & {\@1} \\ FORM & {\it pass} \]\] \\VAL & \[SPR & \< \avml {\@2}\[AGR & {\@1}\]  \avmr \> \\ COMPS & \< \avml {\@B} \avmr \> \] \\ ARG-ST & \< \avml {\@2}X_i \avmr \> \oplus {\@B}\< \avml (\[PP \\ FORM & by \\ INDEX & j \])\avmr \> \\SEM & \[INDEX & {\it s} \\RESTR & \< \avml \[RELN & {\it assume} \\ SIT & s \\ASSUMER & {\it j} \\ ASSUMPTION & {\it i}  \]  \avmr \>\] \]\avmr \>\end{avm}
\end{equation}
\begin{equation}
\begin{avm} \< \avml  assumed , \[{\it word}  \\SYN & \[HEAD & \[{\it verb} \\ AGR & {\@1} \\ FORM & {\it pass} \]\] \\VAL & \[SPR & \< \avml {\@2}NP\[FORM & it\]  \avmr \> \\ COMPS & {\@B}\] \\ ARG-ST & \< \avml{\@2} \avmr \> \oplus {\@B}\< \avml (\[PP \\ FORM & by \\ INDEX & j \])  CP_i \[AGR & {\@1}\]\avmr \> \\SEM & \[INDEX & {\it s} \\RESTR & \< \avml \[RELN & {\it assume} \\ SIT & s \\ASSUMER & {\it j} \\ ASSUMPTION & {\it i}  \]  \avmr \>\] \]\avmr \>\end{avm}
\end{equation}
\subsection{Tree value for i.}
\scalebox{.6}{
\begin{turn}{0}
\begin{forest}
[S \\ \begin{avm} \[VAL & \[SPR & \< \avml \avmr \> \\ COMPS \< \avml  \avmr \> \] \] \end{avm} [NP \\ \begin{avm} {\@1} \[VAL & \[SPR & \< \avml \avmr \> \\ COMPS \< \avml  \avmr \> \] \] \end{avm} [It]] [VP \\ \begin{avm} \[VAL & \[SPR & \< \avml {\@1} \avmr \> \\ COMPS \< \avml  \avmr \> \] \] \end{avm} [V \\ \begin{avm} \[VAL & \[SPR & \< \avml {\@1} \avmr \> \\ COMPS \< \avml {\@2}   \avmr \> \] \] \end{avm} [was]] [VP \begin{avm}{\@2} \end{avm} [V \\ \begin{avm} \[VAL & \[SPR & \< \avml {\@1} \avmr \> \\ COMPS \< \avml {\@3} \avmr \> \] \] \end{avm} [assumed]] [CP \\ \begin{avm} {\@3} \[VAL & \[SPR & \< \avml \avmr \> \\ COMPS \< \avml  \avmr \> \] \] \end{avm} [C \\ \begin{avm} \[VAL & \[SPR & \< \avml \avmr \> \\ COMPS \< \avml {\@4} \avmr \> \] \] \end{avm} [that]] [S \\ \begin{avm} {\@4}\[VAL & \[SPR & \< \avml \avmr \> \\ COMPS \< \avml  \avmr \> \] \] \end{avm} [NP \\ \begin{avm} {\@5} \[VAL & \[SPR & \< \avml \avmr \> \\ COMPS \< \avml  \avmr \> \] \] \end{avm} [D \\ \begin{avm} {\@6}\[VAL & \[SPR & \< \avml \avmr \> \\ COMPS \< \avml  \avmr \> \] \] \end{avm} [the]][NP \\ \begin{avm} \[VAL & \[SPR & \< \avml {\@6} \avmr \> \\ COMPS \< \avml  \avmr \> \] \] \end{avm} [ad]]] [VP \\ \begin{avm} \[VAL & \[SPR & \< \avml {\@5} \avmr \> \\ COMPS \< \avml  \avmr \> \] \] \end{avm} [worked]]]]]]]
\end{forest} \end{turn} }
\section{Chapter 12, Problem 1}
\subsection{Tend}
Tend is a raising verb. \\
i) Yes. \\ There Tends to be problems\\. *There tends to problems.\\
ii) Yes. \\  It tends to rain. \\ *It tends to be rain.\\
iii) Yes. \\ Tabs tended to be kept on Chris.\\ Tabs were kept on Chris. \\
iv) Yes. \\ She tended to talk to Chris. \\Chris tended to be talked to by her.
\subsection{Decide}
Decide is a control verb. \\
i) No. \\  *There decided to be problems. \\
ii) No. \\  *it decided to rain. \\
iii) No. \\  *Tabs were decide to be kept on Chris. \\
iv) No. \\  She decided to talk to Chris.\\ *Chris was decided to be talked to by her. \\
\subsection{Manage}
Manage is a raising verb.\\
i) Yes. \\ There manage to be problems. *There manage to problems \\
ii) Yes. \\ It managed to rain. \\ *It managed to be rain. \\
iii) Yes. \\ Tabs were managed to be kept on Chris. \\ *Tabs were managed to be sought on Chris. \\
iv) Yes. \\ She managed to talk to Chris. \\ Chris was managed to be talked to by her. \\ This last one gave me a pause but then I actually heard two people talk like this during my work week referring to time allowed Chris to be talked to by her. 
\subsection{Fail}
Fail is a raising verb. \\
i) Yes. \\ There fails to be problems. *There fails to problems. \\
ii) Yes. \\ It fails to rain.\\ *It fails to be rain. \\
iii) Yes. \\ Tabs failed to be kept on Chris. \\ *Tabs failed to be sought on Chris. \\
iv) Yes. \\ Chris failed the test. The test was failed by Chris. \\
\subsection{Happen}
Happen is a raising verb. \\
i) Yes. \\ There happens to be problems.\\ * There happens to problems. \\
ii) Yes. \\ It happens to rain. \\ *It happens to be rain. \\
iii) Yes. \\ Tabs happen to be kept on Chris. \\ *Tabs happen to sought on Chris. \\
iv) Yes. \\ She happened to talk to Chris. \\ Chris happened to be talked to by her. 
\end{document}
