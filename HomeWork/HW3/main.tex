\documentclass{article}
\usepackage[utf8]{inputenc}
\usepackage{pgfplots,multicol}
\usepackage{tikz-qtree}
\usepackage{url}
\usepackage{hyperref}
\usepackage{xcolor}
\usepackage{avm}
\usepackage{rtrees}
\usepackage{forest}
\useforestlibrary{linguistics}
\forestapplylibrarydefaults{linguistics}
\hypersetup{
  colorlinks   = true, %Colours links instead of ugly boxes
  urlcolor     = red, %Colour for external hyperlinks
  linkcolor    = blue, %Colour of internal links
  citecolor   = blue %Colour of citations
}
\pgfplotsset{compat=newest} 
\mathchardef\period=\mathcode`.
\DeclareMathSymbol{.}{\mathord}{letters}{"3B}
\newcommand{\textarray}[1]{\ensuremath{\left[ \mbox{\ttfamily\begin{tabular}{l} #1 \end{tabular}}\right]}}
\begin{document}
\title{566 HW3}
\author{Daniel Campos  \tt {dacampos@uw.edu}}
\date{10/18/2019}
\maketitle 
\section{Chapter 6, Problem 2}
\subsection{Lexical Entry pequeños}
Based on the data, it seems that pequeños only is possible for 3rd person. I am unsure if this is the case for the language at large but based my lexical entry based on it.\\
\begin{avm}
\< pequeños ,\[{\it word} \\ SYN \[
	    HEAD \[ {\it adj}\\  AGR & {\@1}\[{\it plural} \\ PER & 3rd \\NUM & {\it pl} \\ GEND & {\it masc} \] \] \\
	    VAL \[ SPR \< \avml \hfil  \avmr \> \\ COMPS \< \avml\hfil \avmr \> \\ MOD \< \avml NP_{i} \[AGR & {\@1}  \] \avmr \> \] \]\\
        SEM \[ MODE & none \\ INDEX & s \\ RESTR \< \avml\hfil \[ RELN  & {\it small} \\ ARG & i \\ SIT & s\] \avmr \> \] \] \>
\end{avm}
\subsection{Tree for the NP los pinguinos pequeños}
\scalebox{.5}{
\begin{forest}
[ \begin{avm}\[{\it phrase} \\ SEM \[ MODE & none \\ INDEX & j \\ RESTR \< \avml {\@3} $.$ {\@4} $.$ {\@5}  \avmr \> \] \] \end{avm} 
    [ \begin{avm}{\@1} \[{\it phrase} \\ SEM \[ MODE & ref \\ INDEX & j \\ RESTR \< \avml{\@3} $.$ {\@4}  \avmr \> \] \] \end{avm}     
        [\begin{avm}{\@2} \[{\it word} \\ SEM \[ MODE & none \\ INDEX & j \\ RESTR \< \avml{\@3} \[RELN & the \\ BV & j\]  \avmr \> \] \] \end{avm} [los]]
        [\begin{avm}\[{\it word} \\ SYN \[VAL \[ SPR \< \avml {\@2} \avmr \> \] \]\\ SEM \[ MODE & ref \\ INDEX & j \\ RESTR \< \avml {\@4}\[RELN & pinguinos \\ INST & j \]  \avmr \> \] \] \end{avm} [pinguinos]]]
    [\begin{avm} \[ {\it word} \\ SYN \[ VAL \[ MOD \< \avml {\@1} NP_{j} \avmr \> \] \] \\ SEM \[ MODE & none \\INDEX & s\\ RESTR \< \avml {\@5}\[RELN & small \\ SIT & s\\ARG & j\]  \avmr \> \] \] \end{avm} [pequeños]]]
\end{forest}}
\subsection{Explain how the INDEX value of pinguinos is identified with the argument of the predication introduced by pequeños.}
The index value of pinguinos is identified by the argument of the predication from pequeños since the lexical entry for pequeños includes a MOD specification that identifies the index of the NP that it is modifying. Additionally, it also gives a predication in its RESTR ({\@5} in our tree above) list which introduces a semantic restriction that in which the index for pequeños must be the same as the noun phrase it is modifying. As the Head Modifier Rule licenses the NP to the top of the tree which as a result  connects the lexical entry for pequeños as a valid modifier for the NP ({\@1} in our tree above) and confirms that the value of the index identified in the MOD requirements of pequeños is the same as the index in the NP. Finally, the Semantic Inheritance principle passes the INDEX value from pinguinos up to the NP and thus the INDEX must be the same as the predication introduced by pequeños in order for the sentence to be licensed.
\section{Chapter 6, Problem 3}
\subsection{Where do the ’s of possession appears in English}
Based on the samples it appears they appear in the last noun of the head noun-phrase of the sentence. In other words right before the noun that is being possessed. 
\subsection{Which of these sentences does CASE predict should be grammatical, and why?}
Given our assumption that CASE is a HEAD feature then CASE will license sentences where the possessed noun is the head of the noun phrase which allows the CASE to follow the Head Feature Principle. In other words, the possessive is being added to the head of the noun phrase. This means it  predicts i, iii, iv, vii to be grammatical. \\
It licenses i because the head of the noun phrase is Leslie but would not be the head of the noun phrase Leslie's coffee. We predict ii to be ungrammatical since university will not be the head of the phrase 'the president of the university' which wouldn't make the np possessive. By the same logic since in iii the possessive is on president this sentence is licensed. Example iv is considered grammatical since trail is the head of its noun phrase. It licenses iv since trail is the head of its own NP which can posses leaves. By the same logic, vii is licensed as grammatical since the noun with the possessive (person) is the head of the noun phrase. 
\section{Chapter 6, Problem 5}
\subsection{SEM value of the determiner ’s?}
\begin{avm}
\< 's , SEM \[ MODE & none \\ INDEX & i  \\ RESTR \< \avml\hfil \[ RELN  & {\it the} \\ BV & i \] \[ RELN  & {\it poss} \\ POSSESSOR & j \\ POSSESSED & i \]\avmr \> \] \>
\end{avm}
\subsection{Tree for the phrase Pat’s book}
\scalebox{.5}{
\begin{forest}
[ NP_{i} \\ \begin{avm}\[ SEM \[ MODE & ref \\ INDEX & i \\ RESTR \< \avml{\@4} $.$ {\@5} $.$ {\@6} $.$ {\@7} \avmr \> \] \] \end{avm} 
    [ DP_{i} \\ \begin{avm}{\@1} \[ SEM \[ MODE & none \\ INDEX & i \\ RESTR \< \avml {\@4} $.$ {\@5} $.$ {\@6}  \avmr \> \] \] \end{avm} 
        [ NP_{j} \\ \begin{avm} {\@2} \[ SEM \[ MODE & ref \\ INDEX & j \\ RESTR \< \avml {\@6} \[ RELN & name \\ NAME & Pat \\ NAMED & j \]  \avmr \> \] \] \end{avm} [Pat]]
        [ D_{i} \\ \begin{avm} \[ SYN \[ VAL \[ SPR \< \avml {\@2}NP \avmr \> \] \] \\ SEM \[ MODE & none \\ INDEX & i \\ RESTR \< \avml {\@4}\[RELN & poss \\ POSSESSOR & j \\ POSSESED & i \] {\@5}\[RELN & the \\ BV & i\] \avmr \> \] \] \end{avm} ['s]]]
    [ N_{i} \\ \begin{avm}\[SYN \[VAL \[ SPR \< \avml {\@2}DetP_{i}  \avmr \> \\ COMPS \< \avml\hfil \avmr \> \\ MOD \< \avml \hfil \avmr \> \] \]\\ SEM \[ MODE & ref \\ INDEX & i \\ RESTR \< \avml {\@7}\[RELN & book \\ INDEX & i\]  \avmr \> \] \] \end{avm} [book]]]
\end{forest}}
\subsection{Guarantee right SEM value for the phrase}
The Semantic Compositionality Principle, the Semantic Inheritance Principle and the Coordination rule ensure that we have the right SEM value for the phrase. \\ We use the Semantic Compositionality Principle since it ensures the mothers RESTR values must be the sum of the RESTR values of its daughters. This means that we have to set up our phrase in such a way that we each mother in the phrase can leverage the RESTR values in the lexicon and ensures that the final SEM RESPR is well structured and fully represented. \\
We leverage the semantic inheritance principle since it dictates that in a headed phrase the mothers mode and index values are identical to its head daughter. This ensures we pass up the correct MODE and index from the lexical entry of Book that license the correct structure for the phrase. \\
Finally, we leverage the coordination rule as a way on ensuring that the Compositionality principle is implemented and ensuring that the sentences INDEX values are well formed to represent the desired semantics. \\
Since these aforementioned principles and rules license up from the lexical entries directly we ensure that our SEM representation for the phrase represents exactly what the joint representation that the individual entries allow. 
\end{document}
