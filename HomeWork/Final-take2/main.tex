\documentclass[12pt]{article}
\usepackage{longtable}
\usepackage[utf8]{inputenc}
\usepackage{pgfplots,multicol}
\usepackage{longtable}
\usepackage{tikz-qtree}
\usepackage{url}
\usepackage{hyperref}
\usepackage{xcolor}
\usepackage{avm}
\usepackage{rtrees}
\usepackage{forest}
\usepackage[utf8]{inputenc}
\usepackage{fourier} 
\usepackage{array}
\usepackage{makecell}

\renewcommand\theadalign{bc}
\renewcommand\theadfont{\bfseries}
\renewcommand\theadgape{\Gape[4pt]}
\renewcommand\cellgape{\Gape[4pt]}
\useforestlibrary{linguistics}
\newcommand{\comment}[1]{}
\newcommand{\myavm}[5]{
    \begin{avm}
        \ifboolexpr{
            test{\ifstrempty{#2}}
            and
            test{\ifstrempty{#3}}
            and
            test{\ifstrempty{#4}}
            and
            test{\ifstrempty{#5}}
        }
        {#1}
        {
            \avml
            \hfil #1 \hfil \\
            \[
                \ifstrempty{#2}{}{SPR       & \< #2 \> \\}
                \ifstrempty{#3}{}{COMPS     & \< #3 \> \\}
                \ifstrempty{#4}{}{GAP       & \< #4 \> \\}
                \ifstrempty{#5}{}{STOP-GAP  & \< #5 \> \\}
            \]
            \avmr
        }
    \end{avm}
}
\forestapplylibrarydefaults{linguistics}
\hypersetup{
  colorlinks   = true, %Colours links instead of ugly boxes
  urlcolor     = red, %Colour for external hyperlinks
  linkcolor    = blue, %Colour of internal links
  citecolor   = blue %Colour of citations
}
\pgfplotsset{compat=newest} 
\mathchardef\period=\mathcode`.
\DeclareMathSymbol{.}{\mathord}{letters}{"3B}
\newcommand{\textarray}[1]{\ensuremath{\left[ \mbox{\ttfamily\begin{tabular}{l} #1 \end{tabular}}\right]}}
\begin{document}
\title{566 Final }
\author{Daniel Campos  \tt {dacampos@uw.edu}}
\date{12/17/2020}
\maketitle 
\section{Problem 1}
Those formulas, scientists expect to be simple to derive and to apply.
\subsection{Tree}
\scalebox{.5}{
\begin{forest}
[ \myavm{S}{}{}{}{} 
    [ \myavm{{\@2}$NP_i$}{}{}{}{} 
        [ \myavm{{\@1}DP}{}{}{}{} 
            [Those] 
        ] 
        [ \myavm{$N_i$}{{\@1}}{}{}{} 
            [formulas]
        ]
    ]  
    [ \myavm{S}{}{}{\@2}{\@2} 
        [  \myavm{{\@3}NP}{}{}{}{} 
            [  \myavm{{}N}{DP}{}{}{} 
                [scientist]
            ]
        ]
        [\myavm{{}VP}{\@3}{\@2}{}{}
            [ \myavm{{}V}{\@3}{\@4}{\@2}{}
                [expect]
            ]
            [ \myavm{{\@4}VP}{\@2}{}{}{} 
                [ \myavm{{}V}{\@2}{\@5}{}{} 
                    [to]
                ]
                [ \myavm{{\@5}VP}{\@2}{}{}{}
                    [ \myavm{{}V}{\@2}{\@6}{}{} 
                        [be]
                    ]
                    [ \myavm{{\@6}AP}{\@2}{}{}{}
                        [ \myavm{{}A}{$\@2_i$}{\@7}{}{$\@8_i$}
                            [simple]
                        ]
                        [ \myavm{{\@7}VP}{\@9}{}{$\@8_i$}{}
                            [ \myavm{{}VP}{\@9}{}{$\@8_i$}{}
                                [ \myavm{{}V}{\@9}{\@{10}}{}{} 
                                    [to]
                                ]
                                [ \myavm{{\@{10}}V}{\@9 NP}{}{$\@8_i$}{}
                                    [derive]
                                ]
                            ]
                       [\myavm{{}CONJ}{}{}{}{}
                                [and]
                            ]
                        [ \myavm{{}VP}{\@9}{}{$\@8_i$}{}
                            [ \myavm{{}V}{\@9}{\@{11}}{}{}
                                    [to]
                            ]
                            [ \myavm{{\@{11}}V}{\@9 NP}{}{$\@8_i$}{}
                                    [apply]
                                ]
                            ]
                        
                        ]
                    
                    ]
                ]  
            ]
        ]
    ]
]
\end{forest} }
\subsection{Identity Cascades}
To save space I will use the following abbreviations:
Head Compliment Rule(HCR), Head Specifier Rule(HSR), Semantic Inheritance Principle(SIP), Argument Realization principle(ARP), Valence Principle(VP), Head Filler Rule(HFR), Coordination Rule (CR) \\
\\ \\ \\
\begin{tiny}
\begin{footnotesize}
\begin{longtable}{|l|l|l|}
\hline
Grammar entity & 1st member of identity & 2nd member of identity \\ \hline
1. Lexical entry for formula & The INDEX value of N {\it formulas}  & the INST value of the {\it formulas} predication\\ \hline
2. SIP & The INDEX value of N {\it formulas}  & \makecell{The INDEX value of the NP \\ dominating {\it those formulas}} \\ \hline
3. HFR & The NP Dominating {\it those formulas} & \makecell{The sole element on the GAP \\list of the S dominating \\{\it scientists expect to } \\{\it be simple to derive and to apply}}  \\ \hline
4. GAP Principle & \makecell{The sole element on the \\ GAP list of the S \\ dominating {\it scientists expect to} \\{\it be simple to derive and to apply}} & \makecell{The sole element on the GAP list of the \\ VP dominating {\it expect to be} \\{\it simple to derive and to apply}} \\ \hline
5. GAP Principle & \makecell{The sole element on the GAP list of \\ the VP dominating {\it expect to} \\{\it be simple to derive and to apply}}  & \makecell{ The sole element on the \\ GAP list of the V {\it expect}} \\ \hline
6. ARP & \makecell{the sole element on the GAP \\ list for the V \\{\it expect}} & \makecell{The second value on the ARG-ST\\ for the V {\it expect}} \\ \hline
7. Lexical entry for {\it expect} & \makecell{The second value on the \\ARG-ST for the V {\it expect}} & \makecell{The sole element on \\ the SPR list of the 3rd element \\ on the ARG-ST for the V {\it expect}} \\ \hline
8. HCR & \makecell{The sole element on the COMPS\\ list of the V {\it expect}} & \makecell{The VP dominating\\ {\it to be simple to} \\{\it derive and to apply}} \\ \hline
9. VP and HCR & \makecell{The sole element on \\ the SPR list of the VP \\ dominating {\it to be simple} \\{\it to derive and to apply}}  & \makecell{The sole element on \\ the SPR list of \\ the V {\it to}} \\ \hline
10. ARP &   \makecell{The sole element on the SPR \\ list of the V\\ {\it to}} &  \makecell{The first element on the\\ ARG-ST of the V {\it to}} \\ \hline
11. Lexical entry for to&   \makecell{The first element on the ARG-ST\\ of the V {\it to}} &  \makecell{The sole element on \\the SPR list of the 2nd element\\ on the ARG-ST of the V {\it to}} \\ \hline
12. ARP &  \makecell{The second element on \\the ARG-ST of the V\\ {\it to}} &  \makecell{The sole element \\on the COMPS list \\of the V {\it to}}\\ \hline
13. HCR &  \makecell{The sole element on the \\ COMPS list of the \\ V {\it to}} &  \makecell{The VP dominating \\{\it be simple to} {\it derive and apply}} \\ \hline
14. VP and HCR &  \makecell{The sole element on \\ the SPR list of the VP \\ dominating {\it be simple} \\{\it to derive and to apply}} &  \makecell{The sole element on the \\ SPR list of the \\ V {\it be}} \\ \hline
15. ARP &  \makecell{The sole element on the SPR\\ list of the V\\ {\it be}} &  \makecell{The first element on \\ the ARG-ST of the \\ V {\it be}} \\ \hline
16. Lexical entry for be &  \makecell{The first element on the \\V {\it be}} &  \makecell{The sole element on the\\ SPR list of the 2nd \\element on the ARG-ST of \\the V {\it be} } \\ \hline
17. ARP &  \makecell{The second element on the \\ ARG-ST of the V \\{\it be}} &  \makecell{The sole element on the \\ comps list of the V {\it be}} \\ \hline
18. HCR &  \makecell{The sole element on the COMPS \\ list of the V \\{\it be}} &  \makecell{The AP dominating \\ {\it simple to} \\{\it derive and to apply}}  \\ \hline
19. GAP Principle & \makecell{The sole element on the GAP list \\ of the AP dominating\\ {\it simple to derive and to apply}} & \makecell{The sole element on the \\ STOP-GAP list of the A dominating {\it simple}} \\ \hline
19. VP and HCR &  \makecell{The sole element on  \\ the SPR list of the \\ A dominating {\it simple}} &  \makecell{The sole element on the SPR\\ list of the a {\it simple}} \\ \hline
20. ARP & \makecell{The sole element on the \\ SPR list of the a\\ {\it simple}} & \makecell{The index of the first \\element on the ARG-ST of \\the A {\it simple}}  \\ \hline
21. Lexical Entry for simple & \makecell{The index of the first element\\ on the ARG-ST of the \\ A {\it simple}} & \makecell{The index of the 1st element on \\ the GAP list of the 2nd element \\ of the ARG-ST of the A {\it simple}} \\ \hline
22. ARP & \makecell{The 2nd element of the ARG-ST \\ of the A {\it simple}} &  \makecell{The sole element \\ on the COMPS list of A {\it simple}} \\ \hline
23. HCR & \makecell{The sole element on the \\ COMPS list of A {\it simple}} & \makecell{The VP  dominating\\ {\it to derive and to apply}} \\ \hline
24. GAP Principle & \makecell{The sole element on the GAP list \\ of the VP dominating\\ {\it to derive and to apply}} & \makecell{The sole element on the \\ GAP list of the VP dominating {\it to derive}} \\ \hline
25. GAP Principle & \makecell{The sole element on the \\ GAP list of the VP dominating \\ {\it to derive}} & \makecell{The sole element \\on the GAP list of the V \\dominating {\it derive}} \\ \hline
26. ARP & \makecell{The sole element of \\the GAP list of the \\V {\it derive}} & \makecell{The second element \\of the ARG-ST of the\\ V {\it derive}} \\ \hline
27. Lexical entry for derive & \makecell{INDEX value of the second \\ element of the ARG-ST of the \\ V derive} & \makecell{DERIVED value of the derive predication} \\ \hline




\end{longtable}
\end{footnotesize}
\end{tiny}
\section{Problem 2}
We hoped that it would surprise Leslie that Kim sang, but it didn't
\subsection{Lexical rules}
\begin{longtable}{|l|l|l|}
\hline
Word & 1st lexical Rule & 2nd lexical rule (if appropriate)  \\ \hline
We & Constant Lexeme Lexical Rule &\\ \hline
hoped & Passive Lexical Rule & Past-Tense Verb Lexical Rule \\ \hline
that & Constant Lexeme Lexical Rule & \\ \hline
it & Constant Lexeme Lexical Rule & \\ \hline
would & Passive Lexical Rule & \\ \hline
surprise & Base Form Lexical Rule & \\ \hline
Leslie & Constant Lexeme Lexical Rule& \\ \hline
that &Constant Lexeme Lexical Rule & \\ \hline
Kim &Constant Lexeme Lexical Rule & \\ \hline
Sang & Past Tense Verb Lexical Rule & Extraposition Lexical Rule \\ \hline
but & Constant Lexeme Lexical Rule & \\ \hline
it &  Constant Lexeme Lexical Rule& \\ \hline
didn't & Non-3rd-Singular Verb Lexical rule & contraction lexical rule \\ \hline
\end{longtable}

\subsection{Word Structure for didn't}
\begin{avm}
 \[{\it word} \\SYN & \[ HEAD & \[{\it verb} \\ FORM & {\it fin} \\ AUX & +\\ PRED & - \\ INF & - \\ POL & + \\ AGR & \@2{\it non-3sing}\] \\ VAL & \[SPR & \< \avml {\@1}\avmr \>  \\ COMPS & \<\ {\@3} \avml \hfil \avmr \>  \\ MOD & \< \avml  \hfil \avmr \>\] \\ GAP & \< \avml \hfil \avmr \>\\ STOP-GAP & \< \avml \hfil \avmr \>\]  \\ ARG-ST & \< \avml {\@1}\[HEAD & \[{\it nominal} \\ CASE & {\it nom} \\ AGR & \@2 \]  \\ VAL & \[ SPR & \< \avml \hfil \avmr \>  \\ COMPS & \< \avml \hfil \avmr \> \] \] , {\@3}\[SYN & \[HEAD & \[{\it verb} \\ FORM & base \\ INF & -\] \\ SPR & \< \avml \@1 \avmr \> \\ COMPS & \< \avml \hfil \avmr \>    \] \\ SEM & \[INDEX & s_2}\]  \]   \avmr \> \\ SEM & \[MODE & {\it prop} \\ INDEX & s_3 \\ RESTR & \< \avml \[RELN & \textbf{not}\\ SIT & s_3 \\ ARG & s_1 \] \avmr \> , \< \avml \[RELN & \textbf{did} \\ SIT & s_1 \\ ARG & s_2  \] \avmr \>  \] \\  \]
\end{avm}
\end{document}